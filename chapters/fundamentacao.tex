% O comando \phantomsection é importante para a correta geração de links pelo hyperref.
\phantomsection

% O \chapter* usa um asterisco para que não seja numerado automaticamente, 
% caso seu orientador prefira assim. Se quiser numerado (ex: "Capítulo 2"), 
% use apenas \chapter{Fundamentação Teórica}.
\chapter{Fundamentação Teórica}
\label{ch:fundamentacao-teorica}

% ---
\section{Introdução ao Capítulo}
\label{sec:fund-intro}
% Escreva aqui um parágrafo introdutório que apresenta os objetivos e a estrutura deste capítulo.
% Ex: "Este capítulo estabelece a base conceitual para o presente trabalho, iniciando pela discussão
% sobre o direito à privacidade e o arcabouço legal da LGPD, passando pelas técnicas de
% anonimização de dados clínicos e concluindo com as métricas utilizadas para avaliar o
% balanço entre a proteção da privacidade e a utilidade dos dados para pesquisa."
% ---

% ---
\section{Privacidade e Proteção de Dados na Saúde: O Cenário da LGPD}
\label{sec:fund-lgpd}

\subsection{O Conceito de Privacidade na Era Digital}
\label{subsec:fund-privacidade}
Em uma das primeiras definições de privacidade, Samuel Warren e Louis Brandeis \cite{WarrenBrandeis1890} articularam a existência de um "direito a ser deixado em paz" (\textit{the right to be let alone}). O argumento era de que a proteção de pensamentos, sentimentos e emoções expressos por meio da escrita ou das artes contra plágio ou apropriação física não era um princípio da propriedade privada (sob o qual a propriedade intelectual é baseada), mas sim um direito a "inviolabilidade pessoal", e portanto não somente  as produções intelectuais ou artísticas, mas também as expressões casuais do dia a dia — pensamentos, emoções e ações - deveriam ser protegidas da exposição pública indesejada. Com isso, a privacidade foi estabelecida como um direito universal que não deriva de um contrato ou de uma relação de propriedade, mas sim como um princípio inerente à dignidade humana, lançando as bases para todo o debate futuro sobre o tema.

Décadas depois, já no início da era da computação, em \textit{Privacy and Freedom}, Alan Westin analisou como as novas tecnologias de vigilância e a capacidade de armazenamento e processamento de dados em computadores, criavam uma ameaça inédita à liberdade individual. Diante desse cenário, Westin formulou sua influente definição de privacidade como o direito de um indivíduo de determinar quando, como e em que medida as informações sobre si são comunicadas a terceiros \cite{Westin1967}. Essa noção, conhecida como "autodeterminação informativa", deslocou o foco de um direito passivo ao isolamento para um poder ativo de controle sobre o fluxo de dados pessoais. É este o princípio que constitui a base da Lei Geral de Proteção de Dados Pessoais (LGPD) brasileira \cite{Brasil2018lgpd}. A LGPD, ao estabelecer as bases legais para o tratamento de dados, como o consentimento informado e a finalidade legítima, materializa a visão de Westin: ela não proíbe o uso de dados, mas o regulamenta para garantir que o titular mantenha o controle sobre suas informações, transformando um princípio filosófico em um direito legal e aplicável.

Outra definição fundamental para este trabalho, que avança em relação à noção de controle individual, é a de Integridade Contextual, proposta por Helen Nissenbaum. Em sua obra \textit{Privacy in Context}, a autora argumenta que a privacidade não é simplesmente sobre manter informações secretas, mas sim sobre garantir que o fluxo de informações pessoais siga as normas esperadas para um determinado contexto social \cite{Nissenbaum2009}. A violação da privacidade ocorre quando essas normas são quebradas. Nissenbaum modela essas normas a partir de três parâmetros: os atores (quem envia, quem recebe), os atributos (o tipo de informação) e os princípios de transmissão (as regras que governam o fluxo). O contexto da saúde pode ser um exemplo: um paciente (ator) espera que seu diagnóstico (atributo) seja compartilhado com outro especialista (ator) sob um princípio de confidencialidade, mas ficaria chocado se o mesmo dado fosse vendido a uma empresa de marketing, pois isso violaria as normas contextuais.

% Para a presente pesquisa, o conceito de Nissenbaum é crucial, pois justifica a anonimização não como uma perda de controle, mas como uma mudança de contexto: ao anonimizar os dados clínicos, removemos os atores originais (pacientes) para adequar a informação a um novo contexto — o da pesquisa científica —, com novos atores (pesquisadores) e novos princípios de transmissão (uso para o bem comum), mantendo assim a integridade do fluxo informacional.

Finalmente, a discussão sobre privacidade na era digital seria incompleta sem analisar a lógica econômica que a desafia em sua essência. Em sua obra \textit{A Era do Capitalismo de Vigilância}, a economista Shoshana Zuboff argumenta que a massiva coleta de dados por empresas de tecnologia não é um simples efeito colateral, mas a fundação de uma nova forma de mercado \cite{Zuboff2019}. Nesse modelo, as experiências humanas são tratadas como matéria-prima gratuita e processadas como "excedente comportamental". Este excedente é utilizado para fabricar "produtos de predição", que são vendidos em novos mercados com o objetivo de antecipar e até mesmo influenciar o comportamento dos usuários. Essa lógica econômica choca-se frontalmente com as definições anteriores: ela torna o ideal de controle individual de Westin uma ilusão, ao operar por meio da extração massiva e da ofuscação, e viola sistematicamente a integridade contextual de Nissenbaum, ao remover os dados de seu contexto original para alimentar outros mercados de futuros comportamentais. Com essa nova lógica mercadológica, a questão da privacidade se transforma de uma violação individual para um desafio sistêmico ao qual a legislação precisa se adaptar. A resposta legal começa por categorizar e definir precisamente o objeto de proteção, o que nos leva à distinção fundamental entre o dado pessoal e o dado sensível.

\subsection{O Dado Pessoal e o Dado Sensível}
\label{subsec:fund-dados-sensiveis}

O Art. 5º, inciso I, da LGPD \cite{Brasil2018lgpd} define dado pessoal como "informação relacionada a pessoa natural identificada ou identificável". Essa definição é propositalmente ampla, abrangendo não apenas dados que apontam diretamente para um indivíduo, como nome completo ou CPF, mas também informações que, isoladamente ou em combinação com outras, podem levar à identificação de uma pessoa. Exemplos incluem endereços IP, dados de geolocalização, ou mesmo uma combinação de atributos como profissão, cidade e idade.

No inciso seguinte do mesmo artigo (Art. 5º, II), a LGPD qualifica o dado sensível, um subconjunto especial de dados pessoais que, devido à sua natureza, requer um nível mais elevado de proteção. Entre as informações destacadas como dados sensíveis pela legislação, estão "dados referentes à saúde" e "dados genéticos ou biométricos". A inclusão desses tipos de dados na categoria de sensíveis reflete o reconhecimento de que sua exposição pode levar a discriminação, estigmatização ou outros danos significativos ao titular. Por exemplo, a divulgação não autorizada de um diagnóstico médico pode afetar a vida pessoal e profissional de um indivíduo, enquanto dados genéticos podem revelar predisposições a certas doenças que podem impactar decisões de seguro ou emprego.

A definição de dado sensível como posta na legislação (por meio de exemplificação) não é exaustiva, e que para determinar um dado como sensível é essencial verificar o contexto de sua utilização e sua relação com outras informações disponíveis. Dessa forma, deve-se admitir que outros dados, não explicitamente mencionados como sensíveis, podem ser considerados como tal a depender do uso que se faz deles e sua potencialidade de se tornar instrumentos de discriminação ou violação de direitos fundamentais \cite{TefferViola2020}. Isso reforça a necessidade de um cuidado redobrado no tratamento de dados pessoais na área da saúde, que mesmo quando não categorizados formalmente como sensíveis, podem revelar aspectos íntimos da vida dos pacientes.

\subsection{Pilares da LGPD: Consentimento e Transparência}
\label{subsec:fund-consentimento}

O Art. 7º da LGPD é dedicado às hipóteses que autorizam o tratamento de dados pessoais. Entre essas, o inciso I destaca tutela especial para o consentimento do titular dos dados - mesmo que esta não seja a única base legal para o uso de dados. O consentimento, conforme definido no Art. 5º, inciso IX, deve ser fornecido de forma livre, informada e inequívoca, garantindo que o titular compreenda plenamente as implicações do tratamento de seus dados. Isso inclui a finalidade específica para a qual os dados serão utilizados, o período de armazenamento e os direitos do titular em relação aos seus dados. Dessa forma, entende-se que o consentimento é restritivo, de modo que o agente não pode estender a autorização concedida para outras finalidades não previstas inicialmente, e que o titular pode revogar o consentimento a qualquer momento, conforme previsto no Art. 8º da LGPD.

Para \citeonline{TefferViola2020}, \textit{Livre} significa que o titular pode escolher a utilização de seus dados sem intervenções ou pressões externas que viciem o consentimento por meio de assimetria entre as partes. \textit{Informado} implica que o titular deve ter a sua disposição informações suficientes e acessíveis para avaliar a forma como seus dados serão tratados e os riscos e implicações do processo, levando em conta a assimetria técnica e informacional entre as partes. Já \textit{inequívoco} sugere que o consentimento deve ser expresso de maneira clara, sem ambiguidades, seja por meio de uma ação afirmativa ou declaração explícita, de modo que o ônus da prova de que o consentimento foi obtido de forma adequada recai sobre o agente de tratamento.

O cumprimento desses requisitos granulares, somado à necessidade de gerenciar a revogação do consentimento, impõe um significativo ônus operacional e burocrático às organizações. Essa definição rigorosa reflete a preocupação da LGPD em garantir que o titular mantenha o controle sobre seus dados pessoais, alinhando-se com o conceito de autodeterminação informativa de Westin discutida na seção anterior.

Apesar de que o consentimento não seja a única base legal para o tratamento de dados, ele é especialmente relevante no contexto de dados sensíveis, que requerem tutela especial. O Art. 11 da LGPD estabelece que o tratamento de dados sensíveis somente poderá ocorrer com o consentimento específico e destacado do titular, salvo em situações excepcionais previstas na lei, como cumprimento de obrigação legal ou regulatória, proteção da vida ou da incolumidade física do titular ou de terceiros, ou para a tutela da saúde, exclusivamente em procedimento realizado por profissionais de saúde, serviços de saúde ou autoridade sanitária.
Essas exceções reconhecem que há circunstâncias em que o tratamento de dados sensíveis é necessário para proteger interesses públicos ou direitos fundamentais, mesmo sem o consentimento do titular.

Como complemento ao consentimento, a LGPD estabelece o princípio da transparência como um de seus pilares fundamentais, conforme o Art. 6º, inciso VI. Este princípio garante aos titulares o acesso a "informações claras, precisas e facilmente acessíveis" sobre como seus dados são tratados. Na prática, a transparência é a ferramenta que viabiliza o consentimento verdadeiramente "informado". Sem que as organizações comuniquem abertamente quais dados são coletados, para qual finalidade e por quanto tempo, o titular não possui as condições necessárias para consentir de forma livre e consciente. Portanto, a combinação do consentimento (a ação do titular) com a transparência (o dever do controlador) busca reequilibrar a relação assimétrica entre as partes, fomentando a confiança no tratamento de dados pessoais.

\subsection{O Tratamento de Dados para Fins de Pesquisa na LGPD}
\label{subsec:fund-pesquisa-lgpd}

Como destacado na seção anterior, o tratamento de dados pessoais, como os dados clínicos, é fortemente regulado pela LGPD, exigindo o consentimento explícito do titular. No entanto, a lei também reconhece a importância do avanço científico e tecnológico para a sociedade, estabelecendo exceções específicas para o uso de dados pessoais sem consentimento em determinados contextos de pesquisa. Em seu Art. 7º, inciso IV, a lei estabelece uma base legal específica que autoriza o tratamento de dados pessoais "para a realização de estudos por órgão de pesquisa", dispensando a necessidade de outras bases legais como o consentimento \cite{Brasil2018lgpd}. Conforme o Art. 5º, XVIII, a definição de "órgão de pesquisa" é ampla, incluindo entidades da administração pública e organizações privadas sem fins lucrativos que tenham como missão institucional a pesquisa básica ou aplicada.

A LGPD é ainda mais específica ao tratar do uso de dados sensíveis no contexto da pesquisa científica. O Art. 11, inciso II, prevê que o tratamento de dados sensíveis pode ocorrer sem o consentimento do titular "para a realização de estudos por órgão de pesquisa, garantida, sempre que possível, a anonimização dos dados pessoais sensíveis" \cite{Brasil2018lgpd}. Essa disposição reconhece que a pesquisa científica muitas vezes depende do acesso a dados sensíveis para gerar conhecimento que pode beneficiar a sociedade como um todo. Ainda assim, a lei estabelece uma diretriz clara que preza pela desvinculação completa entre os dados e os indivíduos, buscando minimizar os riscos de violação de privacidade.

A anonimização dos dados, como especificada na LGPD, é a principal ferramenta para equilibrar a proteção da privacidade dos titulares dos dados com a necessidade de acesso a informações sensíveis para fins de pesquisa e desenvolvimento científico. Essa salvaguarda impões também um desafio técnico significativo para as instituições de pesquisa, que devem ser mais cautelares na forma como coletam, armazenam e processam dados sensíveis a fim de garantir uma anonimização que reduza o risco de reidentificação dos indivíduos ao mesmo tempo em que preserva a utilidade dos dados para análise científica.

\subsection{Anonimização vs. Pseudonimização sob a Ótica da Lei}
\label{subsec:fund-anon-pseudo}
% Apresente a definição técnica e jurídica de cada termo, deixando claro por que a
% anonimização efetiva é o objetivo para dispensar o consentimento.

Tratando-se de um instrumento essencial para a proteção da privacidade, a anonimização é um conceito que ganha atenção especial na LGPD. Conforme o Art. 5º, inciso XI, é definida como a "utilização de meios técnicos razoáveis e disponíveis no momento do tratamento, por meio dos quais um dado perde a possibilidade de associação, direta ou indireta, a um indivíduo". Ainda, a lei também faz a distinção com a pseudonimização, definida como um tratamento de segurança que, por meio da separação de informações, faz com que um dado perca a possibilidade de associação direta a um indivíduo, exceto pelo uso de informação adicional mantida separadamente pelo controlador \cite[Art. 13, § 4º]{Brasil2018lgpd}. Nesse sentido, determina-se que apesar de ambas as técnicas visarem a proteção da privacidade, a anonimização é um processo mais robusto, pois remove permanentemente a possibilidade de reidentificação, enquanto a pseudonimização apenas dificulta essa associação, mas não a elimina completamente.

Como consequência dessa distinção, entende-se que o dado pseudoanonimizado é identificável, e portanto continua sendo considerado um dado pessoal sob a LGPD cujo tratamento requer uma base legal, como o consentimento do titular. Já o dado anonimizado, por perder a possibilidade de associação a um indivíduo, deixa de ser considerado um dado pessoal e, portanto, pode ser tratado sem a necessidade de consentimento, desde que a anonimização seja efetiva e irreversível. Isso significa que, uma vez anonimizado, o conjunto de dados pode ser utilizado para pesquisa e outras finalidades sem as restrições impostas pela LGPD. Para o escopo deste trabalho, o foco será na anonimização, buscando um método de tratamento de dados que dissocie de forma irreversível os dados clínicos de seus titulares.

% ---
\section{O Valor dos Dados Clínicos para Pesquisa e Inovação}
\label{sec:fund-valor-dados}

\subsection{A Importância do Big Data na Saúde}
\label{subsec:fund-bigdata}
% Apresente exemplos de como grandes volumes de dados de saúde podem acelerar descobertas,
% melhorar diagnósticos e otimizar a gestão em saúde pública.

\subsection{Dados Estruturados vs. Dados Não Estruturados}
\label{subsec:fund-dados-estruturados}
% Defina e exemplifique os dois tipos de dados no contexto clínico (ex: exames laboratoriais vs. notas de evolução).
% Discuta os desafios e as oportunidades de cada um para a anonimização e análise.
% ---

% ---
\section{Métodos e Técnicas de Anonimização de Dados}
\label{sec:fund-metodos-anon}

\subsection{Abordagens Iniciais: Generalização e Supressão (k-Anonimato)}
\label{subsec:fund-k-anonimato}
% Explique o conceito de k-Anonimato, como ele funciona na prática (agrupando indivíduos) e
% discuta suas principais vulnerabilidades, como o ataque de homogeneidade.

\subsection{Refinando a Proteção: l-Diversidade e t-Proximidade}
\label{subsec:fund-l-diversidade}
% Descreva como a l-Diversidade e a t-Proximidade surgiram para corrigir as falhas
% do k-Anonimato, adicionando requisitos sobre a variedade dos dados sensíveis nos grupos.

\subsection{Técnicas de Randomização e Perturbação}
\label{subsec:fund-randomizacao}
% Apresente brevemente outras técnicas, como a adição de ruído aleatório, e discuta
% seus prós e contras em relação à utilidade dos dados.

\subsection{Privacidade Diferencial: O Padrão-Ouro de Garantia Matemática}
\label{subsec:fund-privacidade-diferencial}
% Explique de forma conceitual o que é a Privacidade Diferencial. Foque na ideia de que a
% saída de uma consulta é praticamente a mesma com ou sem os dados de um indivíduo específico.
% Mencione o papel do parâmetro epsilon ($\epsilon$) no controle do balanço privacidade-utilidade.
% Lembre-se de usar o modo matemático para o epsilon, assim: $\epsilon$.
% ---

% ---
\section{Métricas de Avaliação: O Balanço entre Privacidade e Utilidade}
\label{sec:fund-metricas}

\subsection{Avaliação do Risco de Reidentificação (Métricas de Privacidade)}
\label{subsec:fund-metricas-privacidade}
% Descreva como se pode quantificar o risco de um indivíduo ser reidentificado em um dataset
% anonimizado. Apresente algumas métricas ou modelos de risco.

\subsection{Avaliação da Utilidade dos Dados Anonimizados (Métricas de Qualidade)}
\label{subsec:fund-metricas-utilidade}
% Detalhe as duas abordagens principais para medir a qualidade/utilidade dos dados:
% 1. Similaridade estatística (comparação de distribuições, correlações, etc.).
% 2. Performance em tarefas de Machine Learning (comparando a performance de um modelo
%    treinado nos dados originais vs. nos dados anonimizados).
% ---

% ---
\section{Síntese da Fundamentação e Escolha da Abordagem}
\label{sec:fund-sintese}
% Feche o capítulo com um ou dois parágrafos que resumam o que foi apresentado e, o mais importante,
% justifique quais técnicas e métricas discutidas aqui serão empregadas na parte prática do seu TCC.
% Isso cria a ponte perfeita para o seu próximo capítulo (Metodologia/Desenvolvimento).
% ---